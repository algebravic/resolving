\documentclass{article}
\usepackage{amsmath}
\usepackage{amsfonts}
\usepackage{amsthm}
\newtheorem{definition}{Definition}
\newtheorem{lemma}{Lemma}
\newtheorem{proposition}{Proposition}
\title{The Metric Dimension of the Hamming Graph}
\author{Victor S. Miller}
\DeclareMathOperator{\Aut}{Aut}
\newcommand{\RR}{\mathbb{R}}
\begin{document}
\maketitle
\begin{abstract}
  The \emph{metric dimension} of a graph is a measure of
  how easy it is to distinguish vertices given distance information.
  It is the smallest cardinality of a subset of vertices so that the set
  of distances from those nodes uniquely determines any other vertex.
  Finding the metric dimension is an NP-complete problem.
  Nevertheless it is of some interest to determine its exact value for
  various families of graphs.  In this note we discuss the use of
  \emph{MaxSat} solvers in determining the metric dimension of the
  \emph{hypercube}.  Since these graphs have a large symmetry group it
  is of great practical importance to make use of the these symmetries
  to speed up the computation.
\end{abstract}

\section{Introduction}
\label{sec:intro}

The concept of \emph{metric dimension} was introduced, independently,
by Harary and Melter \cite{harary1976metric}, and Slater
\cite{slater1975leaves} as an attempt to generalize the continuous
concept of dimension to discrete spaces.  They noted that in the real
Euclidean space $\RR^n$, that there is a set of $n+1$ points, so that
every point in $\RR^n$ is uniquely determined by its distances from
those points, and that $n+1$ is the minimum possible value for the
size of such a set.  In fact the set may be taken to be the 0 vector,
along with the $n$ vectors which have exactly one non-zero coordinate
equal to 1.  Given any metric space one may define the same concept.
In particular, a finite undirected connected graph is a metric space
with the respect to the shortest distance between two vertices in the
graph.

Since the original papers, there have been a large number of works
concerning metric dimension of various graphs.  On one hand, it is
shown that the general problem of computing the metric dimension of a
graph is NP-complete \cite[GT61]{garey1979computers}
\cite{khuller1996landmarks,diaz2012complexity,hauptmann2012approximation}.
In this section we define some basic terminology: \emph{resolving set}
and \emph{metric dimension}.

\begin{definition}
  Let $G$ be a finite graph, and $u \ne v \in V(G)$ two distinct
  vertices of $G$.  A vertex $x \in V(G)$ \emph{resolves} the pair
  $(u,v)$ if $d(x,u) \ne d(x,v)$ or $d(u,x) \ne d(v,x)$, where
  $d(a,b)$ denotes the length of the shortest path from $a$ to $b$.
  A subset of vertices $S \subseteq V(G)$ is a \emph{resolving set}
  for $G$ if every pair of distinct vertices of $G$ is resolved by
  some element of $S$.  The \emph{metric dimension} of $G$ is the
  cardinality of the smallest resolving set for $G$.
\end{definition}
The metric dimension of $G$ is usually denoted by $\beta(G)$, although
some authors use the notation $\mu(G)$.

The problem which we consider here is to calculate the metric
dimension of the \emph{Hypercube}.
This problem was first described by Harary and Melter in
\cite{harary1976metric}. 

\begin{definition}
  Let $S$ be a finite set, and $x,y \in S^n$ be $n$-tuples of elements
  of $S$.  The \emph{Hamming distance} $d_H(x,y)$ is the number of
  indices, $i \in \{1, \dots, n\}$ such that $x_i \ne y_i$.
\end{definition}

\begin{definition}
  Let $n$ be a positive integer. The $n$-dimensional \emph{hypercube}
  $Q^n$. is the undirected graph whose vertices are the set
  $\{0,1\}^n$ of $n$-tuples of 0/1.  Two such vertices are connected
  by an edge if and only if the Hamming distance between the vertices
  is 1.
\end{definition}
Note that if $x,y \in V(Q^n)$ are two vertices then $d_{Q^n}(x,y) =
d_H(x,y)$, the Hamming distance.
\begin{definition}[Weight]
  Let $S$ be a finite set with a distinguished element, denoted by 0.
  The \emph{weight} of an $n$-tuple of elements of $S$, $x$ is the number
  of coordinates of $x$ which are not 0.  Denote the weight of $x$ by $w(x)$.
\end{definition}

If $x,y \in V(Q^n)$ then $d_{Q^n}(x,y) = w(x \oplus y)$, where $x
\oplus y$ is the coordinatewise sum of the elements of $x$ and $y$
taken modulo 2.  If $x * y$ denotes the coordinatewise product of $x$
and $y$, then we have $w(x \oplus y) = w(x) + w(y) - 2 w(x * y)$.

\begin{definition}[Automorphism Group]
  Let $G$ be a finite undirected graph.  An \emph{automorphism}, of
  $G$ is a one-to-one map $\phi: V(G) \rightarrow V(G)$ with the
  property that if $(x,y) \in E(G)$ then
  $(\phi(x), \phi(y)) \in E(G)$.  The set of automorphisms forms a
  group under composition, and is denoted by $\Aut(G)$.  If $G$ has
  the property that given $x, y \in V(G)$ there is an automorphism,
  $\phi$ such that $\phi(x) = y$, then $G$ is said to be \emph{vertex
    transitive}.
  
\end{definition}

\begin{definition}[Hyperoctahedral Group]
The \emph{hyperoctahedral group} of dimension $n$ is (in concrete
form) the set of all maps $\{0,1\}^n \rightarrow \{0,1\}^n$ of the
form $(x,\sigma)$ where $x \in \{0,1\}^n$ and $\sigma \in S_n$,
permutations of $\{1, \dots, n\}$, where $(x, \sigma) (y) = x \oplus
(y_{\sigma(1)}, \dots, y_{\sigma(n)})$.
  
\end{definition}
Note: The hyperoctahedral group of dimension $n$ is $\Aut(Q^n)$.

\begin{definition}[Distance Transitive]
Let $G$ be a finite undirected graph.  It is \emph{distance
  transitive} if given $x,y,z,w \in V(G)$ with $d_G(x,y) = d_G(z,w)$
there is an automorphism $\phi \in \Aut(G)$ such that $\phi(x) = z, \phi(y) = w$.
  
\end{definition}
Note that the hypercube $Q^n$ is distance transitive

Lemma: 
\bibliography{references}
\bibliographystyle{alpha}
\end{document}
