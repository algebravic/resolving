\documentclass{beamer}

\usepackage{amsfonts}
\usepackage{amsmath}
\usepackage{xcolor}
\usepackage{amssymb}
\usepackage{amsthm}
\usepackage{hyperref}
\newcommand{\RR}{\mathbb{R}}
\newcommand{\FF}{\mathbb{F}}
\usetheme{AnnArbor}
\setbeamertemplate{footline}[frame number]
\title{Resolving the Hypercube}
\author{Victor S. Miller}
\institute{SRI International, Computer Science Laboratory \\
Menlo Park, CA USA}
\date{27 June 2023}
\begin{document}
\begin{frame}[plain]
  \maketitle
\end{frame}
\sectionmark{Metric Dimension}
\begin{frame}
  \frametitle{GPS}
  \begin{itemize}
  \item How does GPS work?
  \item Measure the distance from two (or more) satellites.
  \item Uniquely determines location (up to error).
  \item General: How many distance measurements in $n$ dimensions are
    necessary to determine position?
  \item In general $n+1$ (for GPS we know we're on the earth).
  \end{itemize}
\end{frame}
\begin{frame}
  \frametitle{Discrete Analogue: Resolving Sets}
  \begin{itemize}
  \item Introduced by Slater, and Harary and Melter.
  \item Yields a concept of dimension for finite graphs.
  \item Applications
    \begin{itemize}
    \item Coin-weighing.
    \item Network discovery and verification.
    \item Multiuser coding and CDMA.
    \item Compressed Genotyping.
    \item Group Testing.
    \item Robot Navigation.
    \item Drug Discovery.
    \item Mastermind and Wordle.
    \end{itemize}
  \end{itemize}
\end{frame}
\begin{frame}
  \frametitle{Metric Dimension}
  \begin{itemize}
  \item Like GPS: A vertex \emph{resolves} a pair of vertices if their
    distance to the vertex is different.
  \item Resolving set: some vertex will resolve any pair.
  \end{itemize}
\end{frame}
\begin{frame}
  \frametitle{Resolving and Metric Dimension}
  \begin{itemize}
  \item $G$: graph.
  \item $d_G(x,y)$ - length of shortest path from $x$ to $y$.
  \item $s \in V(G)$ \emph{resolves} $x \ne y \in V(G)$ if $d_G(s,x)
    \ne d_G(s,y)$.
  \item $S \subseteq V(G)$: \emph{resolving set} if every pair of
    nodes is resolved by some element of $S$.
  \item \emph{Metric Dimension}: Smallest size of a resolving set.
  \end{itemize}
\end{frame}
\begin{frame}
  \frametitle{Calculation}
  \begin{itemize}
  \item Formulate as a \emph{minimal hitting set} problem.
  \item $\mathcal{S} : S_1, \dots, S_m \subseteq U$, sets.
  \item \emph{Hitting set} for $\mathcal{S}$ is $X \subset U$
    such that $X \cap S_i \ne \emptyset$ for all $i$.
  \item \emph{Minimal hitting set}: size of smallest hitting set.
  \item For metric dimension: $S_{\{x,y\}} := \{ s : d(s,x) \ne
    d(s,y)\}$.
  \item Minimal hitting set: either ILP or Max Sat.
  \item Issue: There can be $\frac 1 2 \#V(G)^2$ hitting sets.
  \item If $\#G$ is large this approach needs too much memory!
  \item Other problem: \emph{symmetries}.
  \end{itemize}
\end{frame}
\begin{frame}
  \frametitle{Dimension in Metric Spaces}
  \begin{itemize}
  \item $V$: a real vector space.
  \item $s \in V$ \emph{resolves} $x \ne y$ if
    $d(s,x) \ne d(s,y)$.
  \item Smallest $S \subset \RR^n$ resolving all pairs
    has cardinality $n+1$.
  \item Like GPS.
  \item Define dimension in any metric space.
  \item Undirected graphs: $d(x,y)=$ length of shortest path
    from $x$ to $y$.
  \end{itemize}
\end{frame}

\section{Coin Weighing, Detecting, and the Hypercube}
\label{sec:hypercube}
\begin{frame}{Coin Weighing}
  \begin{itemize}
  \item $n$ coins: some real, some counterfeit.
  \item Real coins: $a$ grams, counterfeit: $b < a$ grams.
  \item Can weigh any subset.
  \item What is least number of weighings needed to find all
    counterfeit ($\Delta(n)$)?
  \item Original: S{\"o}derburg and Shapiro.
  \item Erd{\H o}s-Renyi: $\Delta(n) \ge 2(1+\delta) n/\log_2(n+1)$.
  \item Whp: $(1+\delta)(n \log_2 9)/(\log_2 n)$ random subsets will
    work.
  \item Lindstrom: Constructed $2^k-1$ subsets for $n=2^{k-1} k$ which
    suffice.
  \item So $\Delta(n) \sim \frac{n}{\log_4 n}$.
  \end{itemize}
\end{frame}
\begin{frame}{Detecting Subsets}
  \begin{itemize}
  \item Cantor-Mills: Determine a subset of $[n]$ by the cardinality
    of its intersections with $S_1, \dots, S_m$.
  \item What's the least $m$, given $n$?
  \item An $m \times n$ 0/1 matrix $A$ is \emph{detecting} if $Ax$
    uniquely specifies $x$, a 0/1 vector.
  \item Like \emph{compressed sensing}: $A$ undetermined, but, $x$ has
    extra properties.
  \item The same as coin-weighing.
  \item Applications to good CDMA codes.
  \end{itemize}
\end{frame}
\begin{frame}
  \frametitle{The Hypercube}
  \begin{itemize}
  \item Vertices: $n$-tuples of 0/1.
  \item Edges: Nodes differ in one coordinate.
  \item Metric Dimension: Almost the same as coin-weighing.
  \item Exact values were claimed to be known for $n \le 10$.
  \item From the literature nobody proved precise bounds.
  \item Constructions: Lindstrom, Cantor-Mills.
  \item Heuristic search: Calculate good upper bound.
  \end{itemize}
\end{frame}
\begin{frame}{Inequalities}
  \begin{itemize}[<+->]
  \item Let $\beta_n$ denote the Metric dimension if $Q^n$.
  \item Then $\beta_n \le \beta_{n+1} \le b_n + 1$.
  \item If $S$ resolves $Q^{n+1}$, drop the last coordinate to resolve
    $Q^n$.
  \item If $S$ resolves $Q^n$, make $S'$ with $S$ extended by 0, and
    then choose $s \in S$, and add $s || 1$.
  \end{itemize}
\end{frame}
\end{document}
